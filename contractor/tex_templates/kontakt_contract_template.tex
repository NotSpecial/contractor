% This .tex file was created automatically.

((= The latex escape filter was added to all text fields.
    Additionally the companyaddress has thew newline filter applied
    booth choice contains a command, therefore no latex escape is applied

    To make the template as editable as possible, we also do the following:
    Whenever there are several choices for a contract item, i.e. booth choice,
    we put them all into the document and just uncomment one.
    This makes it easy to modify the resulting .tex manually

    For this purpose we now create a uncommenting macro =))
((* macro unc(condition) -*))
((( '' if condition else '%' )))
((*- endmacro *))
\documentclass[kontakt]{amivletter}

% Kontact settings
\fairtitle{((( fairtitle|l )))}
\kontaktpresident{((( president|l )))}

% Sender of cover letters
\setkomavar{signature}{((( sender|l|newline )))}

\begin{document}
((* for data in letterdata *))
% AMIV member responsible for company
\representative{((( data.amivrepresentative|l )))}

% Company info
\name{((( data.companyname|l )))}
\address{((( data.companyaddress|l|newline )))}
\city{((( data.companycity|l )))}
\country{((( data.companycountry|l )))}
\contact{((( data.companyrepresentative|l)))}

\boothchoice{%
    % Uncomment the booth choice you need
    ((( unc(data.boothchoice.name=='sA1') )))\smallAone
    ((( unc(data.boothchoice.name=='sA2') )))\smallAtwo
    ((( unc(data.boothchoice.name=='sB1') )))\smallBone
    ((( unc(data.boothchoice.name=='sB2') )))\smallBtwo
    ((( unc(data.boothchoice.name=='bA1') )))\bigAone
    ((( unc(data.boothchoice.name=='bA2') )))\bigAtwo
    ((( unc(data.boothchoice.name=='bB1') )))\bigBone
    ((( unc(data.boothchoice.name=='bB2') )))\bigBtwo
    ((( unc(data.boothchoice.name=='su1') )))\startupone
    ((( unc(data.boothchoice.name=='su2') )))\startuptwo
    \hfill
    % Enter the price
    ((( prices[data.boothchoice.name] ))) CHF
    % Uncomment the lines below if you need booth info (either small or big)
    ((( unc(data.boothchoice.size != 'startup') )))\bigbreak
    ((( unc(data.boothchoice.size != 'startup') )))\noindent Standinformationen:
    ((( unc(data.boothchoice.size == 'small') )))\smallboothinfo
    ((( unc(data.boothchoice.size == 'big') )))\bigboothinfo
    ((( unc(data.boothchoice.size != 'startup') )))\bigbreak
    ((( unc(data.boothchoice.size != 'startup')
    )))\noindent Der Aussteller kann einen eigenen Projektständer sowie
    ((( unc(data.boothchoice.size != 'startup')
    ))) anstelle der Stellwand ein eigenes Banner aufstellen, dieses darf
    ((( unc(data.boothchoice.size != 'startup')
    ))) jedoch die Masse des gebuchten Standes nicht überschreiten.
}

\datechoice{%
    % Uncomment the day choice
    ((( unc(data.days=='first'))))Teilnahme ((( days['first']|fulldate )))
    ((( unc(data.days=='second'))))Teilnahme ((( days['second']|fulldate )))
    ((( unc(data.days=='both')
    )))Teilnahme ((( days['first']|shortdate ))) und (((
        days['second']|fulldate )))
    }

\extrachoice{%
    % Uncomment the parts and bigbreaks you need
    ((( unc(data.media)
    )))\noindent Media Paket \hfill ((( prices.media ))) CHF\\
    ((( unc(data.media) )))\mediainfo
    ((( unc(data.media and data.business) )))\bigbreak

    ((( unc(data.business)
    )))\noindent Business Paket \hfill ((( prices.business ))) CHF\\
    ((( unc(data.business) )))\businessinfo
    ((( unc((data.media or data.business) and data.first) )))\bigbreak

    ((( unc(data.first)
    )))\noindent First Paket \hfill ((( prices.first ))) CHF\\
    ((( unc(data.first) )))\firstinfo

    ((( unc( not(data.first or data.media or data.business) )
    )))Nichts gewählt.
}

% Cover letter and contract twice, uncomment as needed
% First the cover letter
((( unc(not contract_only) )))\makecoverletter
% Next the contract (in duplicate if needed)
\makecontract
((( unc(not contract_only) )))\makecontract

((* endfor *))
\end{document}
