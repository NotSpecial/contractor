% This .tex file was created automatically.
%
% To compile it you need the amiv tex data
% On github: https://github.com/NotSpecial/amivtex
% adjust the /amivtexlocation command at the beginning of the document and
% you should be all set.
((= The latex escape filter was added to all text fields.
    Additionally the companyaddress has thew newline filter applied
    booth choice contains a command, therefore no latex escape is applied

    To make the template as editable as possible, we also do the following:
    Whenever there are several choices for a contract item, i.e. booth choice,
    we put them all into the document and just uncomment one.
    This makes it easy to modify the resulting .tex manually 

    For this purpose we now create a uncommenting macro =))
((* macro unc(condition) -*))
((( '' if condition else '%' )))
((*- endmacro *))
\documentclass{scrartcl}

\author{Alexander Dietmüller}

% The path to amivtex files
\newcommand{\amivtexlocation}{((( texpath )))}

% Needed to make \input work from a different directory
\makeatletter
\def\input@path{{\amivtexlocation}}
\makeatother

% User letter package
\usepackage{scrletter}

% Include amiv settings
\usepackage{\amivtexlocation amivdefaults}

% Include KOMA settings
\usepackage{\amivtexlocation komadefaults}

% Include kontakt contract helpers
\usepackage{\amivtexlocation kontaktutils}

% Kontact settings
\renewcommand{\fairtitle}{((( fairtitle|l )))}
\renewcommand{\amivsectionname}{Kommission Kontakt}
\renewcommand{\amivsectionpresident}{((( president|l )))}

% Sender of cover letters
\setkomavar{signature}{((( sender|l )))}

\begin{document}
((* for data in letterdata *))
% AMIV member responsible for company
\renewcommand{\amivrepresentative}{((( data.amivrepresentative|l )))}

% Company info
\renewcommand{\companyname}{((( data.companyname|l )))}
\renewcommand{\companyaddress}{((( data.companyaddress|l|newline )))}
\renewcommand{\companycity}{((( data.companycity|l )))}
\renewcommand{\companycountry}{((( data.companycountry|l )))}
\renewcommand{\companyrepresentative}{((( data.companyrepresentative|l)))}

\renewcommand{\boothchoice}{%
    % Uncomment the booth choice you need
    ((( unc(data.boothchoice=='sA1') )))\smallAone
    ((( unc(data.boothchoice=='sA2') )))\smallAtwo
    ((( unc(data.boothchoice=='sB1') )))\smallBone
    ((( unc(data.boothchoice=='sB2') )))\smallBtwo
    ((( unc(data.boothchoice=='bA1') )))\bigAone
    ((( unc(data.boothchoice=='bA2') )))\bigAtwo
    ((( unc(data.boothchoice=='bB1') )))\bigBone
    ((( unc(data.boothchoice=='bB2') )))\bigBtwo
    ((( unc(data.boothchoice=='su1') )))\startupone
    ((( unc(data.boothchoice=='su2') )))\startuptwo
    \hfill
    % Enter the price
    ((( prices.booths[data.boothchoice] ))) CHF
    % Uncomment the lines below if you need booth info (either small or big)
    ((( unc(data.boothinfo) )))\bigbreak
    ((( unc(data.boothinfo) )))\noindent Standinformationen:
    ((( unc(data.boothinfo=='small') )))\smallboothinfo
    ((( unc(data.boothinfo=='big') )))\bigboothinfo
    ((( unc(data.boothinfo) )))\bigbreak
    ((( unc(data.boothinfo) )))\noindent Der Aussteller kann einen eigenen 
    ((( unc(data.boothinfo) )))Projektständer sowie anstelle der Stellwand ein 
    ((( unc(data.boothinfo) )))eigenes Banner aufstellen, dieses darf jedoch 
    ((( unc(data.boothinfo) )))die Masse des gebuchten Standes nicht 
    ((( unc(data.boothinfo) )))überschreiten.
}

\renewcommand{\datechoice}{%
    ((( days[data.days]|l )))
}

\renewcommand{\extrachoice}{%
    % Uncomment the parts and bigbreaks you need
    ((( unc(data.media) 
    )))\noindent Media Paket \hfill ((( prices.media ))) CHF\\
    ((( unc(data.media) )))\mediainfo
    ((( unc(data.media and data.business) )))\bigbreak
    
    ((( unc(data.business) 
    )))\noindent Business Paket \hfill ((( prices.business ))) CHF\\
    ((( unc(data.business) )))\businessinfo
    ((( unc((data.media or data.business) and data.first) )))\bigbreak
    
    ((( unc(data.first) 
    )))\noindent First Paket \hfill ((( prices.first ))) CHF\\
    ((( unc(data.first) )))\firstinfo
    
    ((( unc(not (True in [data.first, data.media, data.business])) 
    )))Nichts gewählt.   
}

% Cover letter and contract twice, uncomment as needed
% First the cover letter
((( unc(not contract_only) )))\makecoverletter
% Next the contract (in duplicate if needed)
\makecontract
((( unc(not contract_only) )))\makecontract

((* endfor *))
\end{document}
